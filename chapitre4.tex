\chapter{COMPENSATION PAR RETOUR D’ÉTAT AVEC OBSERVATEUR IDENTITÉ ET MINIMAL IDENTITÉ}
\addcontentsline{toc}{chapter}{COMPENSATION PAR RETOUR D’ÉTAT AVEC OBSERVATEUR IDENTITÉ ET MINIMAL IDENTITÉ }

\section{Observateur identité :}

\section{Observateur minimal identité :}

\subsection{construction d'un observateur minimal identité et calcule des gains de cet observateur : }

On a :\\\\
$A=\begin{bmatrix}
0 & 10.5820\\
0 & -3.3333 \\
\end{bmatrix}$\\\\

Séparation de la matrice A :\\
$\quad A_{11}=0$\\

$A_{12}=10.5820$\\
 
$A_{21}=0$\\

$A_{12}=3.3333 $\\\\

et soit : $B_{1}=0$ et $B_{2}=3.5$\\\\
notre nouveau système s'écrit comme:\\\\

\begin{equation}
\left\{\begin{matrix}
\dot{s}(t)=F_{1}s(t)+G_{tild}y(t)+H_{tild}u(t))\\
 x^(t)=s(t)+G_{1}y(t)\\
\end{matrix}\right.
\end{equation}  
 
telque :\\
$G_{tild}=F_{1}G_{1}-G_{1}A_{11}+A_{21}$\\
$H_{tild}=B_{2}G-GB_{1}$\\
et $F_{1}$=$A_{22}-G_{1}A_{12}$\\\\

Dans cette partie on calcule les gains de l’observateur de manière à modifier ses valeurs propres pour que l’observateur soit 4 fois plus rapide que la dynamique de la boucle fermée, et on choisit la valeur propre $P=-2.4$, car on cherche à estimer $x_{2}$, alors on prend le réel de la deuxième valeur propre.\\\\  
après avoir utilisé Matlab \hyperref[section1.2]{(voir Annexe 2)}\label{annexe2} on obtient :\\\\
$G_{1}=0.8190$\\\\

et : $F_{1}=-9.6$\\\\


Calcule De $G_{tild}$ et $H_{tild}$:\\\\

on a :$G_{tild}=F_{1}G_{1}-G_{1}A_{11}+A_{21}$ et $H_{tild}=B_{2}-G_{1}B_{1}$\\\\
après avoir remplacer les valeur numérique, on trouve :\\\\

$G_{tild}=-9.8280$  \quad $H_{tild}=3.5$\\

Alors on obtient notre système comme :\\\\

\begin{equation}
\left\{\begin{matrix}
\dot{s}(t)=-9.6s(t)-9.8280y(t)+3.5u(t))\\
 x^(t)=s(t)+0.8190y(t)\\
\end{matrix}\right.
\end{equation}





