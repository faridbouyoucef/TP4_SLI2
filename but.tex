%Farid
\chapter{BUT DE LA MANIPULATION }
\addcontentsline{toc}{chapter}{BUT DE LA MANIPULATION} 

Cette manipulation se propose de réaliser un asservissement de position en mettant en œuvre les techniques d’espace d’état continu. Le procédé utilisé est la platine "asservissement de position", déjà présentée dans les textes des manipulations de licence EEA ainsi que du module Automatique Système Linéaire Invariant I et dont le schéma de principe est rappelé Figure 1.1.\\\\ 


\begin{center}
\includegraphics[scale=0.5]{schema_platine.png}
\captionof{figure}{\textit{Schématisation de la platine "asservissement de position}}
\label{fig1} 
\end{center}

Les valeurs numériques des coefficients connus sont:\\

 $K_e=10(V.tr^-1)$	 \hfill		$K_s=10(V.tr^-1)$	\hfill		$Kg=0.105(V.s.tr^-1)$   
      