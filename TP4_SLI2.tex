\documentclass[12pt, a4paper, openany]{report}

\def\VersionRapport{1.0}

\usepackage[utf8]{inputenc} % un package
\usepackage[T1]{fontenc}      % un second package
\usepackage[francais]{babel}  % un troisième package
\usepackage{layout}
\usepackage[top=2.7cm, bottom=2.5cm, left=3.5cm, right=3cm]{geometry}
\usepackage{setspace}

\frenchbsetup{StandardLists=true} % à inclure si on utilise \usepackage[french]{babel}
%\usepackage{enumitem}
\usepackage[shortlabels]{enumitem}
\usepackage{amssymb}

\usepackage{color}
\usepackage{listings}
\definecolor{dkgreen}{rgb}{0,0.6,0}
\definecolor{gray}{rgb}{0.5,0.5,0.5}
\definecolor{mauve}{rgb}{0.58,0,0.82}

\lstset{frame=tb,
  language=Matlab,
  aboveskip=3mm,
  belowskip=3mm,
  showstringspaces=false,
  columns=flexible,
  basicstyle={\small\ttfamily},
  keywordstyle=\color{blue},
  commentstyle=\color{dkgreen},
  stringstyle=\color{mauve},
  breaklines=true,
  breakatwhitespace=true,
  tabsize=3,
  breaklines=true,
  morekeywords={matlab2tikz},
  morekeywords=[2]{1}, 
  keywordstyle=[2]{\color{black}},
  identifierstyle=\color{black},
  numbers=left,
  numberstyle={\tiny \color{black}},
  numbersep=9pt, 
  emph=[1]{for,end,break},
  emphstyle=[1]\color{red}
}



\usepackage{multirow} % pour les tableaux
\usepackage[table]{xcolor} % pour les tableaux

\usepackage{verbatim}
\usepackage{moreverb}
\usepackage{url}
\usepackage{pst-all}
\usepackage{eso-pic,graphicx}
\usepackage{caption} 
\usepackage[colorlinks=true,urlcolor=blue,linkcolor=red]{hyperref}
\usepackage{array}
\usepackage[toc,page]{appendix}
\usepackage[off]{auto-pst-pdf}
\usepackage{hyperref} % pour le sommaire table des matières
\AddThinSpaceBeforeFootnotes % à insérer si on utilise \usepackage[french]{babel}
\FrenchFootnotes % à insérer si on utilise \usepackage[french]{babel}
\usepackage{fancyhdr}
\pagestyle{headings}
\usepackage{pifont}  %pour les puces
\usepackage{amsmath} %pour les puces

\usepackage{verbatim} % pour le code en annexe 

%%%%%%%colones 
\newcolumntype{R}[1]{>{\raggedleft\arraybackslash }b{#1}}
\newcolumntype{L}[1]{>{\raggedright\arraybackslash }b{#1}}
\newcolumntype{C}[1]{>{\centering\arraybackslash }b{#1}}
%%%%%%% 

\renewcommand{\appendixpagename}{Annexes}
\renewcommand{\appendixtocname}{Annexes}

\title{Theme: Compte Rendu Système Linéaire à Temps Continu 2}
\author{REBOUT \bsc{Mehenna}}
\author{BOUYOUCEF \bsc{Farid}}
\date{2018-2019}



%new
\newcommand{\HRule}{\rule{\linewidth}{0.5mm}}


\begin{document}

%\selectlanguage{francais}
\pagenumbering{arabic} 

\makeatletter
  \begin{titlepage}
  

  \begin{sffamily}
   \begin{center}

    % Upper part of the page. The '~' is needed because \\
    % only works if a paragraph has started.
    \includegraphics[scale=0.5]{Logo_UT3.jpg}~\\[1.5cm]

    \textsc{\LARGE Master 1 EEA ISTR/RODECO  }\\[2cm]

    \textsc{\Large Compte Rendu  Système Linéaire à Temps Continu 2}\\[1.5cm]

    % Title
    \HRule \\[0.4cm] % saut de ligne
    { \huge \bfseries TP 4 MOTEUR II\\[0.4cm] }

    \HRule \\[1cm]   % sous de ligne 
    \includegraphics[scale=0.1]{logomaster.jpg}
    \\[1cm]

    % Author and supervisor
    \begin{minipage}{0.4\textwidth}
      \begin{flushleft} \large
         \textsc{\emph {Réalisés par:} \\REBOUT Mehenna}\\
         \textsc{BOUYOUCEF Farid}   
          \newline
          Promotion 2018-2019 \\
      \end{flushleft}
    \end{minipage}
    \begin{minipage}{0.4\textwidth}
      \begin{flushright} \large
        \emph{Tuteur :}  \textsc{M LABIT}\\
        \emph{Responsable de la Formation:} \textsc{M GOUAISBAUT}
      \end{flushright}
    \end{minipage}

    \vfill

    % Bottom of the page
    {\large Octobre 2018}

  \end{center}
  \end{sffamily}      
          
  \end{titlepage}
  
\makeatother




   
%*********************** somaire **************
\renewcommand{\contentsname}{Sommaire}
\tableofcontents
%*********************** listes des figures **************
\listoffigures
%*********************** listes des tableaux **************
\listoftables
 
 
 
 %*********************** INTRODUCTION **************
\chapter*{Introduction}
\addcontentsline{toc}{chapter}{Introduction}
************************** a modifier *********** Dans ce TP on va réaliser un asservissement de position ,on va utiliser pour cette manipulation la platine voir   la Figure 1.1.




Les valeurs numériques des coefficients connus sont:\\

 
 %Farid
\chapter{ L'objet du TP }
\addcontentsline{toc}{chapter}{L'objet du TP} 
 
  \section{Objet Du TP}
      Le but de cette manipulation est de réaliser un asservissement de niveau par retour de sortie. En effet, le vecteur d’état du procédé hydraulique n’est pas entièrement accessible par la mesure. On se propose ainsi de mettre en œuvre l’estimation de l’état par un observateur.
  
  
 \chapter{IDENTIFICATION ET ANALYSE DU PROCÉDÉ }
\addcontentsline{toc}{chapter}{IDENTIFICATION ET ANALYSE DU PROCÉDÉ }
 \chapter{MISE EN PLACE D'UN RETOUR D’ÉTAT }
\addcontentsline{toc}{chapter}{MISE EN PLACE D'UN RETOUR D’ÉTAT }

\section{La justification de la possibilité d'annuler l'erreur de position :}




\section{Calcul des gain K et N permettant de placer les valeurs propres du système asservi en $P_{1}$, $P_{2}$ et d'annuler l'erreur statique :}

On a : $P_{1}$, $P_{2}=-2.4\pm5.5j$ \\\\

Pour calculer la valeur du Gain K on pose :\\

$K=[k_{1}\quad k_{2}]$\\\\
et on a : 
$det(\lambda I_{dim2}-(A-BK))=(\lambda-P_{1})(\lambda-P_{2})$\\\\
    
D'ou :\\
$\lambda I_{dim2}-(A-BK)=\begin{bmatrix} 
\lambda & -10.5820  \\
 3.5k_{1} & \lambda+3.3333+3.5k_{2} 
\end{bmatrix} $\\\\
 
après avoir l'identification des deux déterminants on obtient alors :\\

$k_{1}=0.9723  ,\quad k_{2}=0.4190 $, \\\\

D'ou : $K=[0.9723 \quad 0.4190]$\\\\

De même, on a calculer la valeur de K en utilisant MATLAB. \hyperref[section1.2]{(voir Annexe 1)}\label{annexe1}\\\\

ET pour calculer la valeur du Gain N on a :\\\\

$N=1/(C*(-A+B*K)^{-1}*B*\varepsilon $ \quad avec : $\varepsilon=1$, car on veut que l'erreur statique doit être nulle \\\\

et on obtient $N=0.9723$ \quad \hyperref[section1.2]{(voir Annexe 1)}\label{annexe1}\\\\

\section{La représentation d'état de la boucle fermée :}




 \chapter{COMPENSATION PAR RETOUR D’ÉTAT AVEC OBSERVATEUR IDENTITÉ ET MINIMAL IDENTITÉ}
\addcontentsline{toc}{chapter}{COMPENSATION PAR RETOUR D’ÉTAT AVEC OBSERVATEUR IDENTITÉ ET MINIMAL IDENTITÉ }

\section{Observateur identité :}
























\section{Observateur minimal identité :}





 \chapter{COMPENSATION PAR RETOUR DE SORTIE DYNAMIQUE - OBSERVATEUR FONCTIONNEL }
\addcontentsline{toc}{chapter}{COMPENSATION PAR RETOUR DE SORTIE DYNAMIQUE - OBSERVATEUR FONCTIONNEL}


 %%******************* Coclusion
\chapter*{Conclusion}
\addcontentsline{toc}{chapter}{Conclusion}
Ce TP nous a permet de prendre connaissance de la commande d'un système asservi par  retour d'état et du gain de pré-compensation, on as aussi améliorer la précision en régime établis, on aussi déterminer le lieu des pôles du système pour assurer la stabilité.




\begin{appendices}
\chapter*{Annexe 1}

\hyperref[annexe1]{(Retour)}\label{section1.1}
	
	
\begin{lstlisting}
clear all
close all
clc

Ke = 3.6/1000*60/(2*pi);
Ks=10;
Kg=0.105;
Te = 0.01;
Km=10;
Tm=0.3;
Kc=3.5/100;


A=[0 Ks/(9*Kg);0 (-1/Tm)]
B=[0; (Km*Kg)/Tm]
C=[1 0]
D=[0]

sys=ss(A,B,C,D)
Vp=eig(A)

Co=ctrb(sys);
rang_co=rank(Co);

Obs=obsv(sys);
rang_obs=rank(Obs);

K=acker(A,B,[-2.4+5.5*i -2.4-5.5*i])
% K1=0.9723;
% K2=0.0.4190;

N=1/(C*inv(-A+B*K)*B)
% N=0.9723


Abf=A-B*K;
sys=ss(Abf,B,C,D);
Vp1=eig(Abf);
		
			    			
	\end{lstlisting}


\chapter*{Annexe 2}

\hyperref[annexe2]{(Retour)}\label{section1.2}
	
	
\begin{lstlisting}
% ilaq adawigh le scripte de K 

clear all 
close all
clc

S=0.0154;
Sn=5*10^-5;
g=9.81;
H10=0.27474;
H20=0.0299;
H30=0.1368;
H00=0;
a13=0.4753*Sn*sqrt(2*g);
a32=0.4833*Sn*sqrt(2*g);
a20=0.9142*Sn*sqrt(2*g);

R13=(2*sqrt(H10-H30))/a13;
R32=(2*sqrt(H30-H20))/a32;
R20=(2*sqrt(H20-H00))/a20;

A=[-1/(S*R13) 1/(S*R13) 0;
    1/(S*R13) -(1/S)*((1/R13)+(1/R32)) 1/(S*R32);
    0 1/(S*R32) -(1/S)*((1/R32)+(1/R20))]
B=[ 1/S; 0; 0]
C=[1 0 0];
D=[0];

sys=ss(A,B,C,D);
Vp=eig(A);
Co=ctrb(sys)
rang_co=rank(Co);
%le rang=3=dimension du systeme, alors le systeme est commandable
		
			    			
	\end{lstlisting}
				
\end{appendices}





%******  Bibliographie  **********
\bibliographystyle{alpha}
\bibliography{biblio}




\end{document}







